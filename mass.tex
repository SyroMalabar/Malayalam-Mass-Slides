\documentclass[20pt]{beamer}
 
\setbeamercolor{normal text}{bg=black, fg=white}
\setbeamertemplate{navigation symbols}{}
\setbeamertemplate{frametitle}[default][center]
\makeatletter 
\setbeamertemplate{frametitle continuation}{\gdef\beamer@frametitle{}}
\makeatother 
\setbeamerfont{frametitle}{size=\large}

\usepackage{fontspec}
\usepackage{polyglossia}
\setdefaultlanguage{malayalam}
% \setmainfont[Script=Malayalam, HyphenChar="0000]{Rachana}
\setsansfont[Script=Malayalam, HyphenChar="0000]{Rachana}
% In the above line we customized Hyphenation characters since
% visbile hyphen, aka Soft Hyphen is not used for Malayalam
%\lefthyphenmin=3
%\righthyphenmin=4

\title{വിശുദ്ധ കുര്‍ബാന}

\newcommand{\Priest}[1]{\color{white}#1}
\newcommand{\People}[1]{\color{yellow}#1}

\newcommand{\Ammen}{\People{ആമ്മേന്‍.}}

\begin{document}

\begin{frame}
\frametitle{കുരിശടയാളം}
\Priest{
പിതാവിന്റെയും പുത്രന്റെയും പരിശുദ്ധാത്മാവിന്റെയും നാമത്തില്‍.
}
\bigbreak
\Ammen
\end{frame}

\begin{frame}
\frametitle{മിശിഹായുടെ കല്പന}
\Priest{
അന്നാപ്പെസഹാത്തിരുനാളില്‍\\
കര്‍ത്താവരുളിയ കല്പനപോല്‍\\
തിരുനാമത്തില്‍ ചേര്‍ന്നീടാം\\
ഒരുമയോടീ ബലിയര്‍പ്പിക്കാം.
}
\bigbreak
\People{
അനുരഞ്ജിതരായ് തീര്‍ന്നീടാം\\
നവമൊരു പീഠമൊരുക്കീടാം\\
ഗുരുവിന്‍ സ്നേഹമോടീയാഗം\\
തിരുമുബാകെയണച്ചീടാം.
}
\end{frame}

\begin{frame}
\frametitle{മാലാഖമാരുടെ കീര്‍ത്തനം}
\Priest{
അത്യുന്നതങ്ങളില്‍ ദൈവത്തിനു സ്തുതി.
}
\bigbreak
\Ammen
\bigbreak
\Priest{
ഭൂമിയില്‍ മനുഷ്യർക്കു സമാധാനവും പ്രത്യാശയും എപ്പോഴും എന്നേക്കും.
}
\bigbreak
\Ammen
\end{frame}

\begin{frame}
\frametitle{മാലാഖമാരുടെ കീര്‍ത്തനം}
\Priest{
അത്യുന്നതമാം സ്വര്‍ലോകത്തില്‍ സര്‍വ്വേശനു സ്തുതിഗീതം.
}
\bigbreak
\People{
ഭൂമിയിലെങ്ങും മര്‍ത്യനു ശാന്തി പ്രത്യാശയുമെന്നേക്കും.
}
\end{frame}

\begin{frame}[allowframebreaks]
\frametitle{സ്വര്‍ഗ്ഗസ്ഥനായ ഞങ്ങളുടെ പിതാവേ}
\Priest{
സ്വർഗ്ഗസ്ഥനായ ഞങ്ങളുടെ പിതാവേ,
}
\People{
അങ്ങയുടെ നാമം പൂജിതമാകണമേ./ അങ്ങയുടെ രാജ്യം വരണമേ./
അങ്ങയുടെ തിരുമനസ്സു സ്വര്‍ഗ്ഗത്തിലെ പോലെ ഭൂമിയിലുമാകണമേ./
ഞങ്ങള്‍ക്ക് ആവശ്യകമായ ആഹാരം/ ഇന്നു ഞങ്ങള്‍ക്കു തരണമേ.
ഞങ്ങളുടെ കടക്കാരോട് ഞങ്ങള്‍ ക്ഷമിച്ചിരിക്കുന്നത് പോലെ/
ഞങ്ങളുടെ കടങ്ങളും പാപങ്ങളും ഞങ്ങളോടും ക്ഷമിക്കണമേ./
ഞങ്ങളെ പ്രലോഭനത്തില്‍ ഉള്‍പ്പെടുത്തരുതേ/ ദുഷ്ടാരൂപിയില്‍ 
നിന്ന് ഞങ്ങളെ രക്ഷിക്കണമേ/ എന്തുകൊണ്ടെന്നാല്‍/ രാജ്യവും
ശക്തിയും മഹത്ത്വവും/ എന്നേക്കും അങ്ങയുടെതാകുന്നു.\bigbreak ആമ്മേന്‍.
}
\end{frame}

\begin{frame}[allowframebreaks]
\frametitle{സ്വര്‍ഗ്ഗസ്ഥനായ ഞങ്ങളുടെ പിതാവേ}
\Priest{
സ്വര്‍ഗ്ഗസ്ഥിതനാം താതാ നിന്‍\\
നാമം പൂജിതമാകണമേ.\\
നിന്‍ രാജ്യം വന്നീടണമേ\\
പരിശുദ്ധന്‍ നീ പരിശുദ്ധന്‍.\\
}
\bigbreak
\People{
സ്വര്‍ഗ്ഗസ്ഥിതനാം താത നിന്‍\\
സ്തുതിതന്‍ നിസ്തുല മഹിമാവാല്‍\\
ഭൂസ്വര്‍ഗ്ഗങ്ങള്‍ നിറഞ്ഞു സദാ\\
പാവനമായി വിളങ്ങുന്നു.\\
\bigbreak
വാനവമാനവവൃന്ദങ്ങള്‍\\
ഉദ്ഖോഷിപ്പു സാമോദം\\
പരിശുദ്ധന്‍ നീ എന്നെന്നും\\
പരിശുദ്ധന്‍ നീ പരിശുദ്ധന്‍.\\
\bigbreak
സ്വര്‍ഗ്ഗസ്ഥിതനാം താത നിന്‍\\
നാമം പൂജിതമാകണമേ.\\ 
നിന്‍രാജ്യം വന്നീടണമേ\\
നിന്‍ ഹിതമിവിടെഭവിക്കണമേ.\\
\bigbreak
സ്വര്‍ഗ്ഗത്തെന്നതുപോലുലകില്‍\\
നിന്‍ ചിത്തം നിറവേറണമേ.\\
ആവശ്യകമാമാഹാരം\\
ഞങ്ങള്‍ക്കിന്നരുളീടണമേ.\\
\bigbreak
ഞങ്ങള്‍ കടങ്ങള്‍ പൊറുത്തതുപോല്‍\\
ഞങ്ങള്‍ക്കുള്ള കടം സകലം\\ 
പാപത്തിന്‍ കടബാദ്ധ്യതയും\\ 
അങ്ങു കനിഞ്ഞുപൊറുക്കണമേ.\\
\bigbreak
ഞങ്ങള്‍ പരീക്ഷയിലോരുനാളും\\
ഉള്‍പ്പെടുവാനിടയാകരുതേ\\ 
ദുഷ്ടാരൂപിയില്‍ നിന്നെന്നും\\ 
ഞങ്ങളെ രക്ഷിച്ചരുളണമേ.\\
\bigbreak
എന്തെന്നാലെന്നാളേക്കും\\ 
രാജ്യം ശക്തി മഹത്ത്വങ്ങള്‍\\
താവകമല്ലോ കര്‍ത്താവേ\\
ആമ്മേനാമ്മേനെന്നേക്കും.\\
}
\end{frame}
\end{document}
