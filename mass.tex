\documentclass[20pt]{beamer}
 
\setbeamercolor{normal text}{bg=black, fg=white}
\setbeamertemplate{navigation symbols}{}
\setbeamertemplate{frametitle}[default][center]
\makeatletter 
\setbeamertemplate{frametitle continuation}{\gdef\beamer@frametitle{}}
\makeatother 
\setbeamerfont{frametitle}{size=\large}
\makeatletter
\define@key{beamerframe}{c}[true]{% centered
  \beamer@frametopskip=0pt plus 1fill\relax%
  \beamer@framebottomskip=0pt plus 1fill\relax%
  \beamer@frametopskipautobreak=0pt plus .4\paperheight\relax%
  \beamer@framebottomskipautobreak=0pt plus .6\paperheight\relax%
  \def\beamer@initfirstlineunskip{}%
}
\makeatother

\usepackage{fontspec}
\usepackage{polyglossia}
\setdefaultlanguage{malayalam}
% \setmainfont[Script=Malayalam, HyphenChar="0000]{Rachana}
\setsansfont[Script=Malayalam, HyphenChar="0000]{Rachana}
% In the above line we customized Hyphenation characters since
% visbile hyphen, aka Soft Hyphen is not used for Malayalam
%\lefthyphenmin=3
%\righthyphenmin=4

% TODO: Justification ?
%\usepackage{ragged2e}
%\apptocmd{\frame}{}{\justifying}{} % Allow optional arguments after frame.
%\renewcommand{\raggedright}{\leftskip=0pt \rightskip=0pt plus 0cm} 

\setlength{\parskip}{0.5em}

\title{വിശുദ്ധ കുര്‍ബാന}

\newcommand{\Priest}[1]{\color{white}#1}
\newcommand{\People}[1]{\color{yellow}#1}
\newcommand{\Server}[1]{\color{lightgray}#1}
\newcommand{\Notice}[1]{\color{lime}(#1)}

\newcommand{\SignOfCross}{പിതാവും പുത്രനും പരിശുദ്ധാത്മാവുമായ സര്‍വ്വേശ്വരാ, എന്നേക്കും.}
\newcommand{\Ammen}{\People{ആമ്മേന്‍.}}
\newcommand{\Peace}{\Server{സമാധാനം നമ്മോടുകൂടെ.}}
\newcommand{\PPeace}{\Priest{സമാധാനം + നിങ്ങളോടുകൂടെ.}}
\newcommand{\LetsPray}{\Server{നമുക്കൂ പ്രാര്‍ത്ഥിക്കാം,\\ സമാധാനം നമ്മോടുകൂടെ.}}

\newcommand{\AmbaramanavathamPPL}{
\People{
ദിവ്യത്മാവിന്‍ ഗീതികളാല്‍\\
ഹല്ലേലൂയ്യ ഗീതികളാല്‍\\
കര്‍ത്താവിന്‍ തിരുദിവസത്തിന്‍\\
നിര്‍മ്മലമാകുമനുസ്മരണം\\
കൊണ്ടാടാം ഇന്നീ വേദികയില്‍.}}

\begin{document}

\begin{frame}
\frametitle{കുരിശടയാളം}
\Priest{
പിതാവിന്റെയും പുത്രന്റെയും പരിശുദ്ധാത്മാവിന്റെയും നാമത്തില്‍.}

\Ammen
\end{frame}

\begin{frame}
\frametitle{മിശിഹായുടെ കല്പന}
\Priest{
അന്നാപ്പെസഹാത്തിരുനാളില്‍\\
കര്‍ത്താവരുളിയ കല്പനപോല്‍\\
തിരുനാമത്തില്‍ ചേര്‍ന്നീടാം\\
ഒരുമയോടീ ബലിയര്‍പ്പിക്കാം.}\par
\People{
അനുരഞ്ജിതരായ് തീര്‍ന്നീടാം\\
നവമൊരു പീഠമൊരുക്കീടാം\\
ഗുരുവിന്‍ സ്നേഹമോടീയാഗം\\
തിരുമുബാകെയണച്ചീടാം.}
\end{frame}

\begin{frame}
\frametitle{മാലാഖമാരുടെ കീര്‍ത്തനം}
\Priest{
അത്യുന്നതങ്ങളില്‍ ദൈവത്തിനു സ്തുതി.}\par
\Ammen\par
\Priest{
ഭൂമിയില്‍ മനുഷ്യർക്കു സമാധാനവും പ്രത്യാശയും എപ്പോഴും എന്നേക്കും.}\par
\Ammen
\end{frame}

\begin{frame}
\frametitle{മാലാഖമാരുടെ കീര്‍ത്തനം}
\Priest{
അത്യുന്നതമാം സ്വര്‍ലോകത്തില്‍ സര്‍വ്വേശനു സ്തുതിഗീതം.}\par
\People{
ഭൂമിയിലെങ്ങും മര്‍ത്യനു ശാന്തി പ്രത്യാശയുമെന്നേക്കും.}
\end{frame}

\begin{frame}[allowframebreaks]
\frametitle{സ്വര്‍ഗ്ഗസ്ഥനായ ഞങ്ങളുടെ പിതാവേ}
\Priest{
സ്വർഗ്ഗസ്ഥനായ ഞങ്ങളുടെ പിതാവേ,}
\People{
അങ്ങയുടെ നാമം പൂജിതമാകണമേ./ അങ്ങയുടെ രാജ്യം വരണമേ./
അങ്ങയുടെ തിരുമനസ്സു സ്വര്‍ഗ്ഗത്തിലെ പോലെ ഭൂമിയിലുമാകണമേ./
ഞങ്ങള്‍ക്ക് ആവശ്യകമായ ആഹാരം/ ഇന്നു ഞങ്ങള്‍ക്കു തരണമേ.
ഞങ്ങളുടെ കടക്കാരോട് ഞങ്ങള്‍ ക്ഷമിച്ചിരിക്കുന്നത് പോലെ/
ഞങ്ങളുടെ കടങ്ങളും പാപങ്ങളും ഞങ്ങളോടും ക്ഷമിക്കണമേ./
ഞങ്ങളെ പ്രലോഭനത്തില്‍ ഉള്‍പ്പെടുത്തരുതേ/ ദുഷ്ടാരൂപിയില്‍ 
നിന്ന് ഞങ്ങളെ രക്ഷിക്കണമേ/ എന്തുകൊണ്ടെന്നാല്‍/ രാജ്യവും
ശക്തിയും മഹത്ത്വവും/ എന്നേക്കും അങ്ങയുടെതാകുന്നു.\par\Ammen}
\end{frame}

\begin{frame}
\People{
സ്വർഗ്ഗസ്ഥനായ ഞങ്ങളുടെ പിതാവേ,/ അങ്ങയുടെ മഹത്ത്വത്താല്‍/
സ്വര്‍ഗ്ഗവും ഭൂമിയും നിറഞ്ഞിരിക്കുന്നു./ മാലാഖമാരും മനുഷ്യരും/
അങ്ങ് പരിശുദ്ധന്‍, പരിശുദ്ധന്‍ പരിശുദ്ധന്‍ എന്ന് ഉദ്ഘോഷിക്കുന്നു.}
\end{frame}

\begin{frame}[allowframebreaks]
\frametitle{സ്വര്‍ഗ്ഗസ്ഥിതനാം താതാ നിന്‍}
\Priest{
സ്വര്‍ഗ്ഗസ്ഥിതനാം താതാ നിന്‍\\
നാമം പൂജിതമാകണമേ.\\
നിന്‍ രാജ്യം വന്നീടണമേ\\
പരിശുദ്ധന്‍ നീ പരിശുദ്ധന്‍.}\par
\People{
സ്വര്‍ഗ്ഗസ്ഥിതനാം താത നിന്‍\\
സ്തുതിതന്‍ നിസ്തുല മഹിമാവാല്‍\\
ഭൂസ്വര്‍ഗ്ഗങ്ങള്‍ നിറഞ്ഞു സദാ\\
പാവനമായി വിളങ്ങുന്നു.\par
\framebreak
വാനവമാനവവൃന്ദങ്ങള്‍\\
ഉദ്ഖോഷിപ്പു സാമോദം\\
പരിശുദ്ധന്‍ നീ എന്നെന്നും\\
പരിശുദ്ധന്‍ നീ പരിശുദ്ധന്‍.\par
സ്വര്‍ഗ്ഗസ്ഥിതനാം താത നിന്‍\\
നാമം പൂജിതമാകണമേ.\\ 
നിന്‍രാജ്യം വന്നീടണമേ\\
നിന്‍ ഹിതമിവിടെഭവിക്കണമേ.\par
\framebreak
സ്വര്‍ഗ്ഗത്തെന്നതുപോലുലകില്‍\\
നിന്‍ ചിത്തം നിറവേറണമേ.\\
ആവശ്യകമാമാഹാരം\\
ഞങ്ങള്‍ക്കിന്നരുളീടണമേ.\par
ഞങ്ങള്‍ കടങ്ങള്‍ പൊറുത്തതുപോല്‍\\
ഞങ്ങള്‍ക്കുള്ള കടം സകലം\\ 
പാപത്തിന്‍ കടബാദ്ധ്യതയും\\ 
അങ്ങു കനിഞ്ഞുപൊറുക്കണമേ.\par
\framebreak
ഞങ്ങള്‍ പരീക്ഷയിലോരുനാളും\\
ഉള്‍പ്പെടുവാനിടയാകരുതേ\\ 
ദുഷ്ടാരൂപിയില്‍ നിന്നെന്നും\\ 
ഞങ്ങളെ രക്ഷിച്ചരുളണമേ.\par
എന്തെന്നാലെന്നാളേക്കും\\ 
രാജ്യം ശക്തി മഹത്ത്വങ്ങള്‍\\
താവകമല്ലോ കര്‍ത്താവേ\\
ആമ്മേനാമ്മേനെന്നേക്കും.}
\end{frame}

\begin{frame}
\LetsPray
\end{frame}

% Sundays and Ordinary Celebrations
\begin{frame}[allowframebreaks]
\frametitle{പ്രാരംഭ പ്രാര്‍ത്ഥന}
\Priest{
ഞങ്ങളുടെ കര്‍ത്താവായ ദൈവമേ, മനുഷ്യ വംശത്തിന്റെ
നവീകരണ ത്തിനും രക്ഷയ്ക്കും വേണ്ടി അങ്ങയുടെ പ്രിയപുത്രന്‍
കാരുണ്യപൂര്‍വ്വം നല്‍കിയ ദിവ്യരഹസ്യങ്ങളുടെ
പരികര്‍മ്മത്തിനു ബലഹീനരായ ഞങ്ങളെ ശക്തരാക്കണമേ.
സകലത്തിന്‍റെയും നാഥാ, എന്നേക്കും.} \Ammen
\end{frame}

% Other Celebrations
\begin{frame}[allowframebreaks]
\frametitle{പ്രാരംഭ പ്രാര്‍ത്ഥന}
\Priest{
ഞങ്ങളുടെ കര്‍ത്താവായ ദൈവമേ, അങ്ങയുടെ തിരുനാമത്തില്‍ 
ഉറച്ചുവിശ്വസിക്കുകയും ആ വിശ്വാസം പരമാര്‍ത്ഥതയോടെ 
ഏറ്റുപറയുകയും ചെയുന്നവരെ അങ്ങു ശക്തരാക്കണമേ.
ആത്മ ശരീരങ്ങളെ പവിത്രീകരിക്കുന്ന ഈ പരിഹാര രഹസ്യങ്ങള്‍ 
അവര്‍ വിശുദ്ധിയോടെ പരികര്‍മ്മം ചെയ്യട്ടെ. നിര്‍മ്മല
ഹൃദയത്തോടും വിശുദ്ധ വിചാരങ്ങളോടും കൂടെ അവര്‍
അങ്ങേക്കു പുരോഹിതശുശ്രൂഷ ചെയ്യുകയും അങ്ങു കനിഞ്ഞു 
നല്‍കിയ രക്ഷയെ പ്രതി നിരന്തരം അങ്ങയെ സ്തുതിക്കുകയും ചെയ്യട്ടെ.
\SignOfCross} \Ammen
\end{frame}

% Other Celebrations
\begin{frame}[allowframebreaks]
\frametitle{പ്രാരംഭ പ്രാര്‍ത്ഥന}
\Priest{
ഞങ്ങളുടെ കര്‍ത്താവായ ദൈവമേ, അങ്ങയുടെ മഹനീയ
ത്രിത്വത്തിന്റെ സംപൂജ്യമായ നാമത്തിനു സ്വര്‍ഗ്ഗത്തിലും ഭൂമിയിലും
എപ്പോഴും സ്തുതിയും ബഹുമാനവും കൃതജ്ഞതയും ആരാധനയും
ഉണ്ടായിരിക്കട്ടെ. \SignOfCross} \Ammen
\end{frame}

% TODO: add prayers according to kalam

\begin{frame}[allowframebreaks]
\frametitle{സങ്കീര്‍ത്തനം}
\Priest{
കര്‍ത്താവേ, മമ രാജാവേ,\\
പാടും നിൻപുകളെന്നും ഞാൻ\\
സകലേശാ, നിൻ തിരുനാമം\\
വാഴ്ത്തീടും ഞാനനവരതം.}\par
\framebreak
\People{
കര്‍ത്താവേ, നിൻ സ്‌തുതി പാടും\\
അനുദിനമങ്ങയെ വാഴ്ത്തും ഞാൻ\\
നാഥൻ മഹിമ നിറഞ്ഞവനും\\
പാരം സ്തുത്യനുമെന്നെനും.\par
അനവദ്യൻ മമ നാഥാ, നിൻ\\
മഹിമയ്ക്കളവില്ലറിവൂ ഞാൻ\\
തലമുറ തലമുറയോടെന്നും\\
തവ കർമ്മങ്ങൾ വിവരിക്കും.}
\end{frame}

% TODO: add സങ്കീര്‍ത്തനം according to kalam

% not used
\iffalse
\begin{frame}[allowframebreaks]
\frametitle{ധൂപാശീര്‍വ്വാദം}
\Priest{
പിതാവും പുത്രനും പരിശുദ്ധാത്മാവുമായ സര്‍വേശ്വര,
അങ്ങയുടെ ബഹുമാനത്തി നായി ഞങള്‍ സമര്‍പ്പിക്കുന്ന ഈ ധുപം
അങ്ങയുടെ മഹനീയ ത്രിത്വത്തിന്‍റെ നാമത്തില്‍ + ആശീര്‍വ്വദിക്കപ്പെടട്ടെ.
ഇത് അങ്ങയുടെ പ്രസാദത്തിനും അങ്ങയുടെ അജഗണത്തിന്റെ
പാപങ്ങളുടെ മോചനത്തിനും കാരണമാകട്ടെ. എന്നേക്കും.} \Ammen
\end{frame}
\fi

\begin{frame}
\Peace
\end{frame}

% Sundays
\begin{frame}[allowframebreaks]
\frametitle{ഉത്ഥാനഗീതത്തിന് ഒരുക്കം}
\Priest{
ഞങ്ങളുടെ കര്‍ത്താവായ ദൈവമേ, അങ്ങയുടെ സ്നേഹത്തിന്റ
പരിമളം ഞങ്ങളില്‍ വീശുകയും അങ്ങയുടെ സത്യത്തിന്റെ
ജ്ഞാനം ഞങ്ങളുടെ ആത്മാക്കളെ പ്രകാശിപ്പിക്കുകയും
ചെയ്യുമ്പോള്‍ സ്വര്‍ഗ്ഗത്തില്‍നിന്നു പ്രത്യക്ഷനാകുന്ന
അങ്ങയുടെ തിരുക്കുമാരനെ സ്വീകരിക്കുവാന്‍ ഞങ്ങള്‍ക്ക്
ഇടയാകട്ടെ. സകല സൗഭാഗ്യങ്ങളും നന്മകളും നിറഞ്ഞ്
മുടിചൂടി നില്ക്കുന്ന സഭയില്‍ നിരന്തരം അങ്ങയെ സ്തുതിക്കുവാനും
മഹത്വപ്പെടുത്തുവാനും ഞങ്ങള്‍ യോഗ്യരാകട്ടെ.
എന്തുകൊണ്ടെന്നാല്‍, അങ്ങ് എല്ലാത്തിന്റെയും സ്രഷ്ടാവാകുന്നു.
സകലത്തിന്റെയും നാഥാ, എന്നേക്കും.} \Ammen
\end{frame}

% Ordinary Days
\begin{frame}[allowframebreaks]
\frametitle{ഉത്ഥാനഗീതത്തിന് ഒരുക്കം}
\Priest{
ഞങ്ങളുടെ കര്‍ത്താവായ ദൈവമേ, അങ്ങു നല്കിയുട്ടുള്ളതും
എന്നാല്‍ കൃത്യജ്ഞത പ്രകാശിപ്പിക്കുവാന്‍ ഞങ്ങള്‍ക്കു
കഴിയാത്തതുമായ എല്ലാ സഹായങ്ങള്‍ക്കും 
അനുഗ്രഹങ്ങള്‍ക്കും ആയി സകല സൗഭാഗ്യങ്ങളും
നന്മകളും നിറഞ്ഞ് മുടിചൂടി നില്‍ക്കുന്ന സഭയില്‍ ഞങ്ങള്‍
അങ്ങയെ നിരന്തരം സ്തുതിക്കുകയും മഹത്വപ്പെടുത്തുകയും ചെയ്യട്ടെ.
അങ്ങ് സകലത്തിന്റെയും നാഥനും സ്രഷ്ടാവുമാകുന്നു.
\SignOfCross} \Ammen
\end{frame}


\begin{frame}[allowframebreaks]
\frametitle{ഉത്ഥാനഗീതം}
\People{
സര്‍വാധിപനാം കര്‍ത്താവേ,\\
നിന്നെ വണങ്ങി നമിക്കുന്നു.\\
ഈശോനാഥാ, വിനയമോടെ\\
നിന്നെ നമിച്ചു പുകഴ്ത്തുന്നു.\par
മർത്യനു നിത്യമഹോന്നതമാം\\
ഉത്ഥാനം നീയരുളുന്നു.\\
അക്ഷയമാവനുടെ ആത്മാവി-\\
ന്നുത്തമരക്ഷയുമേകുന്നു.}
\end{frame}

\begin{frame}
\LetsPray
\end{frame}

\begin{frame}
\Priest{
എന്റെ കര്‍ത്താവേ, നീ സത്യമായും ഞങ്ങളുടെ ശരീരങ്ങളെ
ഉയര്‍പ്പിക്കുന്നവന്നും ആത്മാക്കളെ രക്ഷിക്കുന്നവനും ജീവനെ
നിത്യം പരിപാലിക്കുന്നവനുമാകുന്നു. ഞങ്ങള്‍ എപ്പോഴും
നിനക്ക് സ്തുതിയും കൃതജ്ഞതയും ആരാധനയും സമര്‍പ്പിക്കുവാന്‍
കടപ്പെട്ടവരാകുന്നു. സകലത്തിന്റെയും നാഥാ, എന്നേക്കും.}
\Ammen
\end{frame}

\begin{frame}
\frametitle{ത്രൈശുദ്ധ കീര്‍ത്തനം}
\People{
ശബ്ദമുയര്‍ത്തിപ്പാടിടുവിന്‍\\
സര്‍വ്വരൂമൊന്നായി പാടിടുവിന്‍\\
എന്നെന്നും ജീവിക്കും\\
സർവ്വേശ്വരനെ വാഴ്ത്തിടുവിന്‍.\par
പരിപാവനനാം സര്‍വ്വേശാ,\\
പരിപാവനനാം ബലവാനെ,\\
പരിപാവനനാം അമര്‍ത്യനെ,\\
നിന്‍കൃപ ഞങ്ങള്‍ക്കേകണമേ.}
\end{frame}

\begin{frame}
\frametitle{വചനശുശ്രൂഷ}
\Priest{
വിശുദ്ധരിൽ സംപ്രീതനായി വസിക്കുന്ന പരിശുദ്ധനും
സ്തുത്യര്‍ഹനും ബലവാനും അമര്‍ത്യനുമായ കര്‍ത്താവേ,
അങ്ങയുടെ സ്വഭാവത്തിനൊത്തവിധം എപ്പോഴും
ഞങ്ങളെ കടാക്ഷിക്കുകയും അനുഗ്രഹിക്കുകയും ഞങ്ങളോടു
കരുണ കാണിക്കുകയും ചെയ്യണമേ. \SignOfCross} \Ammen
\end{frame}

\begin{frame}
\frametitle{പഴയനിയമ വായന}
\Server{
സഹോദരരേ, നിങ്ങള്‍ ഇരുന്നു ശ്രദ്ധയോടെ ദൈവവചനം കേള്‍ക്കുവിന്‍.}\par
\Notice{സമൂഹം ഇരിക്കുന്നു.}\par
\People{ദൈവമായ കര്‍ത്താവിനു സ്തുതി.}
\end{frame}

\begin{frame}[allowframebreaks]
\frametitle{സ്തുതിഗീതം}
\Priest{
അംബരമനവരതം\\
ദൈവമഹത്വത്തെ\\
വാഴ്ത്തിപ്പാടുന്നു.}\par
\AmbaramanavathamPPL\par
\framebreak
\Priest{
തൻ കരവിരുതല്ലോ\\
വാനവിതാനങ്ങൾ\\
ഉദ്ഘോഷിക്കുന്നു.}\par
\AmbaramanavathamPPL\par
\framebreak
\Priest{
പകലുകള്‍ പകലുകളോ-\\
ടവിരതമവിടുത്തെ\\
പുകളുര ചെയ്യുന്നു.}\par
\AmbaramanavathamPPL\par
\framebreak
\Priest{
നിത്യപിതാവിനും\\
സുതനും റൂഹായ്ക്കും\\
സ്തുതിയുണ്ടാകട്ടെ.}\par
\AmbaramanavathamPPL\par
\framebreak
\Priest{
ആദിയിലെപ്പോലെ\\
ഇപ്പോഴുമെപ്പോഴും\\
എന്നേക്കും ആമ്മേന്‍.}\par
\AmbaramanavathamPPL
\end{frame}

\begin{frame}
\Server{ഹല്ലേലൂയ്യ ഹല്ലേലൂയ്യ ഹല്ലേലൂയ്യ;}\par
\LetsPray
\end{frame}

% Sundays
\begin{frame}[allowframebreaks]
\Priest{
 ഞങ്ങളുടെ കര്‍ത്താവായ ദൈവമേ, അങ്ങയുടെ
 ജീവദായകവും ദൈവികവുമായ കല്പനകളുടെ മധുരസ്വരം
 ശ്രവിക്കുന്നതിനും ഗ്രഹിക്കുന്നതിനും ഞങ്ങളുടെ ബുദ്ധിയെ
 പ്രകാശിപ്പിക്കണമേ. അതുവഴി ആത്മശരീരങ്ങള്‍ക്ക്
ഉപകരിക്കുന്ന സ്നേഹവും ശരണവും രക്ഷയും ഞങ്ങളില്‍
ഫലമണിയുന്നതിന്നും നിരന്തരം ഞങ്ങള്‍ അങ്ങയെ
സ്തുതിക്കുന്നതിനും അങ്ങയുടെ കാരുണ്യത്താലും
അനുഗ്രഹത്താലും ഞങ്ങളെ സഹായിക്കണമേ.
\SignOfCross} \Ammen
\end{frame}

% Ordinary Days
\begin{frame}[allowframebreaks]
\Priest{
TODO
} \Ammen
\end{frame}

\begin{frame}
\frametitle{ലേഖന വായന}
\People{നമ്മുടെ കര്‍ത്താവായ മിശിഹായ്ക്കു സ്തുതി.}\par
\Notice{സമൂഹം എഴുന്നേല്‍ക്കുന്നു.}
\end{frame}

\begin{frame}[allowframebreaks]
\frametitle{ഹല്ലേലൂയ്യ ഗീതം}
\People{
ഹല്ലേലൂയ്യ പാടീടുന്നേന്‍\\
ഹല്ലേലൂയ്യ ഹല്ലേലൂയ്യ.\par
നല്ലോരാശയമെന്‍ മനതാരില്‍\\
വന്നു നിറഞ്ഞു തുളുബീടുന്നു.\par
രാജാവിന്‍ തിരുമുമ്പില്‍ കീര്‍ത്തന\\
മധുവായ് ഞാനതൊഴുക്കീടട്ടെ.\\
\framebreak
ഏറ്റമനുഗ്രഹപൂരിതനാം കവി തന്‍ \\
തൂലികപോലെന്‍നാവിപ്പോള്‍.\par
താതനുമതുപോല്‍ സുതനും\\
പരിശുദ്ധാത്മാവിനും സ്തുതിയു യരട്ടെ.\par
ആദിമുതല്ക്കേയിന്നും നിത്യവു-\\
മായി ഭവിചീടട്ടെ, ആമ്മേന്‍.\par
ഹല്ലേലൂയ്യ പാടീടുന്നേന്‍\\
ഹല്ലേലൂയ്യ ഹല്ലേലൂയ്യ.}
\end{frame}

% not used
\iffalse
\begin{frame}[allowframebreaks]
\frametitle{സുവിശേഷവായനയ്ക്ക് ഒരുക്കം}
\Priest{
ലോകത്തിന്‍റെ പ്രകാശവും സകലത്തിന്‍റെയും ജീവനുമായ 
മിശിഹായെ, നിന്നെ ഞങ്ങളുടെ പക്കലേക്കയച്ച 
അനന്തകാരുണ്യത്തിന് എന്നേക്കും സ്തുതി. ആമ്മേന്‍.}
\end{frame}
\fi

\begin{frame}
\frametitle{സുവിശേഷ വായന}
\Server{നമുക്കു ശ്രദ്ധാപൂര്‍വ്വം നിന്ന് പരിശുദ്ധ സുവിശേഷം ശ്രവിക്കാം.}\\~\\
\PPeace\\
\People{അങ്ങയോടും അങ്ങയുടെ ആത്മാവോടും കൂടെ.}\\~\\
\People{നമ്മുടെ കര്‍ത്താവായ മിശിഹായ്ക്കു സ്തുതി.}
\end{frame}

% TODO കാറോസൂസ
\begin{frame}
\frametitle{കാറോസൂസ}
TODO കാറോസൂസ
\end{frame}

\begin{frame}
%\frametitle{കാഴ്ചസമര്‍പ്പണഗീതം} % -> cant fit on one slide
\Server{കര്‍ത്താവില്‍ ഞാന്‍ ദൃഡമായി ശരണപ്പെട്ടു}\\
\People{
മിശിഹാ കര്‍ത്താവിന്‍\\
തിരുമെയ്‌ നിണവുമിതാ\\
പാവന ബലിപീഠേ\\
സ്നേഹഭയങ്ങളോടണയുക നാ-\\
മഖിലരുമോന്നായ് സന്നിധിയില്‍\\
വാനവനിരയോടു ചേര്‍ന്നേവം \\
പാടാം ദൈവം പരിശുദ്ധന്‍\\
പരിശുദ്ധന്‍, നിത്യം പരിശുദ്ധന്‍}
\end{frame}

\begin{frame}[allowframebreaks]
\Priest{
താതനുമതുപോലത്മജനും ദിവ്യറുഹായ്ക്കും സ്തുതിയെന്നും
ദൈവാംബികയാകും കന്യാമറിയത്തെ 
സാദരമോർത്തീടാം പാവനമീബലിയില്‍}\par
\People{
ആദിയിലെപ്പോല്‍ എന്നെന്നേക്കും
ആമ്മേനാമ്മേന്‍.\\
സുതനുടെ പ്രേഷിതരെ, ഏകജസ്നേഹിതരെ,
ശാന്തി ലഭിച്ചിടുവാന്‍ നിങ്ങള്‍ പ്രാര്‍ത്ഥിപ്പിന്‍.}\\
\framebreak
\Priest{
സര്‍വ്വരുമൊന്നായ്‌ പാടീടട്ടെ 
ആമ്മേനാമ്മേന്‍.
മാർത്തോമ്മായേയും, നിണസാക്ഷികളെയും 
സത്ക്കർമ്മികളേയും, ബലിയിതിലോര്‍ത്തീടാം.}\par
\People{
നമ്മുടെകൂടെ ബലവാനാം
കര്‍ത്താവെന്നെന്നേയ്ക്കും.
രാജാവാം ദൈവം, നമ്മോടൊത്തെന്നും 
യാക്കോബിന്‍ ദൈവം, നമ്മുടെ തുണയെന്നും.}\\
\framebreak
\Priest{
ചെറിയവരെല്ലാം വലിയവരൊപ്പം
കാത്തുവസിക്കുന്നു.\\
മൃതരെല്ലാവരും നിൻ, മഹിതോത്ഥാനത്തിൽ\\
ശരണം തേടുന്നു ഉത്ഥിതരായിടുവാന്‍.}\par
\People{
തിരുസ്സന്നിധിയില്‍  ഹൃദയഗതങ്ങള്‍
ചൊരിയുവിനെന്നേക്കും\\
നോമ്പും പ്രാര്‍ത്ഥനയും, പശ്ചാത്താപവുമായ്
ത്രിത്വത്തെ മോദാല്‍, നിത്യം വാഴ്ത്തീടാം.}\\
\end{frame}

\begin{frame}[allowframebreaks]
\frametitle{വിശ്വാസപ്രഖ്യാപനം}
\Priest{
സര്‍വ്വശക്തനും പിതാവുമായ ഏക ദൈവത്തില്‍ ഞങ്ങള്‍ വിശ്വസിക്കുന്നു.}\par
\People{
ദൃശ്യവും അദൃശ്യവുമായ/ സകലത്തിന്റെയും സ്രഷ്ടാവില്‍/
ഞങ്ങള്‍ വിശ്വസിക്കുന്നു./ ദൈവത്തിന്റെ ഏകപുത്രനും/
സകല സൃഷ്ടികള്‍ക്കും മുമ്പുള്ള ആദ്യജാതനും/
യുഗങ്ങള്‍ക്കെല്ലാം മുമ്പ് പിതാവില്‍ \\ \framebreak 
നിന്ന് ജനിച്ചവനും/ എന്നാല്‍ സൃഷ്ട്ടിക്കപെടാത്തവനും/
എകകര്‍ത്താവുമായ ഈശോ/ മിശിഹായില്‍ ഞങ്ങള്‍ വിശ്വസിക്കുന്നു./
അവിടുന്നു സത്യദൈവത്തില്‍ നിന്നുള്ള സത്യദൈവവും/
പിതാവിനോടു കൂടെ ഏകസത്തയുമാകുന്നു./
അവിടുന്നു വഴി പ്രപഞ്ചം സംവിധാനം ചെയ്യപ്പെടുകയും/
എല്ലാം സൃഷ്ടിക്കപ്പെടുകയും ചെയ്തു./ മനുഷ്യരായ നമുക്ക് വേണ്ടിയും/
\\ \framebreak 
നമ്മുടെ രക്ഷയ്ക്കുവേണ്ടിയും/ അവിടുന്നു 
സ്വര്‍ഗ്ഗത്തില്‍ നിന്നിറങ്ങി/ പരിശുദ്ധാത്മാവിനാല്‍/
കന്യകാമറിയത്തില്‍ നിന്ന് ശരീരം സ്വീകരിച്ച്/മനുഷ്യനായിപ്പിറന്നു.
പന്തിയോസ് പീലാത്തോസിന്‍റെ കാലത്ത്
പീഡകള്‍ സഹിക്കുകയും/ സ്ലീവായില്‍ തറയ്ക്കപ്പെട്ടു 
മരിക്കുകയും/ സംസ്കരിക്കപ്പെടുകയും/
എഴുതപ്പെട്ടിരിക്കുന്നതു പോലെ/\\ \framebreak
മൂന്നാം ദിവസം ഉയിര്‍ത്തെഴുന്നേല്ക്കുകയും ചെയ്തു./
അവിടുന്നു സ്വര്‍ഗ്ഗത്തിലേക്ക് എഴുന്നള്ളി/
പിതാവിന്റെ വലതു ഭാഗത്തിരി ക്കുന്നു./
മരിച്ചവരെയും ജീവിക്കുന്നവരെയും വിധിക്കുവാന്‍/
അവിടുന്നു വീണ്ടും വരുവാനിരിക്കുന്നു./
പിതാവില്‍നിന്നും പുത്രനില്‍നിന്നും പുറപ്പെടുന്ന/\\ \framebreak
സത്യാത്മാവും ജീവദാതാവുമായ/
ഏക പരിശുദ്ധാത്മാവിലും ഞങ്ങള്‍ വിശ്വസിക്കുന്നു.
ഏകവും പരിശുദ്ധവും ശ്ലൈഹീകവും സാര്‍വ്വത്രികവുമായ/
സഭയിലും ഞങ്ങള്‍ വിശ്വസിക്കുന്നു./
പാപമോചനത്തിനായുള്ള ഏക മാമ്മോദീസായും/
ശരീരത്തിന്റെ ഉയിര്‍പ്പും/ നിത്യായുസ്സും ഞങ്ങള്‍ ഏറ്റു
പറയുകയും ചെയ്യുന്നു.\par \Ammen}
\end{frame}

\begin{frame}
\Peace
\end{frame}

\end{document}
