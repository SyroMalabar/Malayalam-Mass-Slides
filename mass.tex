\documentclass[20pt]{beamer}
 
\setbeamercolor{normal text}{bg=black, fg=white}
\setbeamertemplate{navigation symbols}{}
\setbeamertemplate{frametitle}[default][center]
\makeatletter 
\setbeamertemplate{frametitle continuation}{\gdef\beamer@frametitle{}}
\makeatother 
\setbeamerfont{frametitle}{size=\large}
\makeatletter
\define@key{beamerframe}{c}[true]{% centered
  \beamer@frametopskip=0pt plus 1fill\relax%
  \beamer@framebottomskip=0pt plus 1fill\relax%
  \beamer@frametopskipautobreak=0pt plus .4\paperheight\relax%
  \beamer@framebottomskipautobreak=0pt plus .6\paperheight\relax%
  \def\beamer@initfirstlineunskip{}%
}
\makeatother

\usepackage{fontspec}
\usepackage{polyglossia}
\setdefaultlanguage{malayalam}
% \setmainfont[Script=Malayalam, HyphenChar="0000]{Rachana}
\setsansfont[Script=Malayalam, HyphenChar="0000]{Rachana}
% In the above line we customized Hyphenation characters since
% visbile hyphen, aka Soft Hyphen is not used for Malayalam
%\lefthyphenmin=3
%\righthyphenmin=4

% TODO: Justification ?
%\usepackage{ragged2e}
%\apptocmd{\frame}{}{\justifying}{} % Allow optional arguments after frame.
%\renewcommand{\raggedright}{\leftskip=0pt \rightskip=0pt plus 0cm} 

\setlength{\parskip}{0.5em}

\title{വിശുദ്ധ കുര്‍ബാന}

\newcommand{\Priest}[1]{\color{white}#1}
\newcommand{\People}[1]{\color{yellow}#1}
\newcommand{\Server}[1]{\color{lightgray}#1}

\newcommand{\SignOfCross}{പിതാവും പുത്രനും പരിശുദ്ധാത്മാവുമായ സര്‍വ്വേശ്വരാ, എന്നേക്കും.}
\newcommand{\Ammen}{\People{ആമ്മേന്‍.}}
\newcommand{\Peace}{\Server{സമാധാനം നമ്മോടുകൂടെ.}}
\newcommand{\LetsPray}{\Server{നമുക്കൂ പ്രാര്‍ത്ഥിക്കാം,\\ സമാധാനം നമ്മോടുകൂടെ.}}

\begin{document}

\begin{frame}
\frametitle{കുരിശടയാളം}
\Priest{
പിതാവിന്റെയും പുത്രന്റെയും പരിശുദ്ധാത്മാവിന്റെയും നാമത്തില്‍.
}

\Ammen
\end{frame}

\begin{frame}
\frametitle{മിശിഹായുടെ കല്പന}
\Priest{
അന്നാപ്പെസഹാത്തിരുനാളില്‍\\
കര്‍ത്താവരുളിയ കല്പനപോല്‍\\
തിരുനാമത്തില്‍ ചേര്‍ന്നീടാം\\
ഒരുമയോടീ ബലിയര്‍പ്പിക്കാം.
}

\People{
അനുരഞ്ജിതരായ് തീര്‍ന്നീടാം\\
നവമൊരു പീഠമൊരുക്കീടാം\\
ഗുരുവിന്‍ സ്നേഹമോടീയാഗം\\
തിരുമുബാകെയണച്ചീടാം.
}
\end{frame}

\begin{frame}
\frametitle{മാലാഖമാരുടെ കീര്‍ത്തനം}
\Priest{
അത്യുന്നതങ്ങളില്‍ ദൈവത്തിനു സ്തുതി.
}

\Ammen

\Priest{
ഭൂമിയില്‍ മനുഷ്യർക്കു സമാധാനവും പ്രത്യാശയും എപ്പോഴും എന്നേക്കും.
}

\Ammen
\end{frame}

\begin{frame}
\frametitle{മാലാഖമാരുടെ കീര്‍ത്തനം}
\Priest{
അത്യുന്നതമാം സ്വര്‍ലോകത്തില്‍ സര്‍വ്വേശനു സ്തുതിഗീതം.
}

\People{
ഭൂമിയിലെങ്ങും മര്‍ത്യനു ശാന്തി പ്രത്യാശയുമെന്നേക്കും.
}
\end{frame}

\begin{frame}[allowframebreaks]
\frametitle{സ്വര്‍ഗ്ഗസ്ഥനായ ഞങ്ങളുടെ പിതാവേ}
\Priest{
സ്വർഗ്ഗസ്ഥനായ ഞങ്ങളുടെ പിതാവേ,
}
\People{
അങ്ങയുടെ നാമം പൂജിതമാകണമേ./ അങ്ങയുടെ രാജ്യം വരണമേ./
അങ്ങയുടെ തിരുമനസ്സു സ്വര്‍ഗ്ഗത്തിലെ പോലെ ഭൂമിയിലുമാകണമേ./
ഞങ്ങള്‍ക്ക് ആവശ്യകമായ ആഹാരം/ ഇന്നു ഞങ്ങള്‍ക്കു തരണമേ.
ഞങ്ങളുടെ കടക്കാരോട് ഞങ്ങള്‍ ക്ഷമിച്ചിരിക്കുന്നത് പോലെ/
ഞങ്ങളുടെ കടങ്ങളും പാപങ്ങളും ഞങ്ങളോടും ക്ഷമിക്കണമേ./
ഞങ്ങളെ പ്രലോഭനത്തില്‍ ഉള്‍പ്പെടുത്തരുതേ/ ദുഷ്ടാരൂപിയില്‍ 
നിന്ന് ഞങ്ങളെ രക്ഷിക്കണമേ/ എന്തുകൊണ്ടെന്നാല്‍/ രാജ്യവും
ശക്തിയും മഹത്ത്വവും/ എന്നേക്കും അങ്ങയുടെതാകുന്നു.

ആമ്മേന്‍.
}
\end{frame}

\begin{frame}
\People{
സ്വർഗ്ഗസ്ഥനായ ഞങ്ങളുടെ പിതാവേ,/ അങ്ങയുടെ മഹത്ത്വത്താല്‍/
സ്വര്‍ഗ്ഗവും ഭൂമിയും നിറഞ്ഞിരിക്കുന്നു./ മാലാഖമാരും മനുഷ്യരും/
അങ്ങ് പരിശുദ്ധന്‍, പരിശുദ്ധന്‍ പരിശുദ്ധന്‍ എന്ന് ഉദ്ഘോഷിക്കുന്നു.
}
\end{frame}

\begin{frame}[allowframebreaks]
\frametitle{സ്വര്‍ഗ്ഗസ്ഥിതനാം താതാ നിന്‍}
\Priest{
സ്വര്‍ഗ്ഗസ്ഥിതനാം താതാ നിന്‍\\
നാമം പൂജിതമാകണമേ.\\
നിന്‍ രാജ്യം വന്നീടണമേ\\
പരിശുദ്ധന്‍ നീ പരിശുദ്ധന്‍.
}

\People{
സ്വര്‍ഗ്ഗസ്ഥിതനാം താത നിന്‍\\
സ്തുതിതന്‍ നിസ്തുല മഹിമാവാല്‍\\
ഭൂസ്വര്‍ഗ്ഗങ്ങള്‍ നിറഞ്ഞു സദാ\\
പാവനമായി വിളങ്ങുന്നു.

\framebreak
വാനവമാനവവൃന്ദങ്ങള്‍\\
ഉദ്ഖോഷിപ്പു സാമോദം\\
പരിശുദ്ധന്‍ നീ എന്നെന്നും\\
പരിശുദ്ധന്‍ നീ പരിശുദ്ധന്‍.

സ്വര്‍ഗ്ഗസ്ഥിതനാം താത നിന്‍\\
നാമം പൂജിതമാകണമേ.\\ 
നിന്‍രാജ്യം വന്നീടണമേ\\
നിന്‍ ഹിതമിവിടെഭവിക്കണമേ.

\framebreak
സ്വര്‍ഗ്ഗത്തെന്നതുപോലുലകില്‍\\
നിന്‍ ചിത്തം നിറവേറണമേ.\\
ആവശ്യകമാമാഹാരം\\
ഞങ്ങള്‍ക്കിന്നരുളീടണമേ.

ഞങ്ങള്‍ കടങ്ങള്‍ പൊറുത്തതുപോല്‍\\
ഞങ്ങള്‍ക്കുള്ള കടം സകലം\\ 
പാപത്തിന്‍ കടബാദ്ധ്യതയും\\ 
അങ്ങു കനിഞ്ഞുപൊറുക്കണമേ.

\framebreak
ഞങ്ങള്‍ പരീക്ഷയിലോരുനാളും\\
ഉള്‍പ്പെടുവാനിടയാകരുതേ\\ 
ദുഷ്ടാരൂപിയില്‍ നിന്നെന്നും\\ 
ഞങ്ങളെ രക്ഷിച്ചരുളണമേ.

എന്തെന്നാലെന്നാളേക്കും\\ 
രാജ്യം ശക്തി മഹത്ത്വങ്ങള്‍\\
താവകമല്ലോ കര്‍ത്താവേ\\
ആമ്മേനാമ്മേനെന്നേക്കും.\\
}
\end{frame}

\begin{frame}
\LetsPray
\end{frame}

% Sundays and Ordinary Celebrations
\begin{frame}[allowframebreaks]
\frametitle{പ്രാരംഭ പ്രാര്‍ത്ഥന}
\Priest{
ഞങ്ങളുടെ കര്‍ത്താവായ ദൈവമേ, മനുഷ്യ വംശത്തിന്റെ
നവീകരണ ത്തിനും രക്ഷയ്ക്കും വേണ്ടി അങ്ങയുടെ പ്രിയപുത്രന്‍
കാരുണ്യപൂര്‍വ്വം നല്‍കിയ ദിവ്യരഹസ്യങ്ങളുടെ
പരികര്‍മ്മത്തിനു ബലഹീനരായ ഞങ്ങളെ ശക്തരാക്കണമേ.
സകലത്തിന്‍റെയും നാഥാ, എന്നേക്കും.
}
\Ammen
\end{frame}

% Other Celebrations
\begin{frame}[allowframebreaks]
\frametitle{പ്രാരംഭ പ്രാര്‍ത്ഥന}
\Priest{
ഞങ്ങളുടെ കര്‍ത്താവായ ദൈവമേ, അങ്ങയുടെ തിരുനാമത്തില്‍ 
ഉറച്ചുവിശ്വസിക്കുകയും ആ വിശ്വാസം പരമാര്‍ത്ഥതയോടെ 
ഏറ്റുപറയുകയും ചെയുന്നവരെ അങ്ങു ശക്തരാക്കണമേ.
ആത്മ ശരീരങ്ങളെ പവിത്രീകരിക്കുന്ന ഈ പരിഹാര രഹസ്യങ്ങള്‍ 
അവര്‍ വിശുദ്ധിയോടെ പരികര്‍മ്മം ചെയ്യട്ടെ. നിര്‍മ്മല
ഹൃദയത്തോടും വിശുദ്ധ വിചാരങ്ങളോടും കൂടെ അവര്‍
അങ്ങേക്കു പുരോഹിതശുശ്രൂഷ ചെയ്യുകയും അങ്ങു കനിഞ്ഞു 
നല്‍കിയ രക്ഷയെ പ്രതി നിരന്തരം അങ്ങയെ സ്തുതിക്കുകയും ചെയ്യട്ടെ.
\SignOfCross
}
\Ammen
\end{frame}

% Other Celebrations
\begin{frame}[allowframebreaks]
\frametitle{പ്രാരംഭ പ്രാര്‍ത്ഥന}
\Priest{
ഞങ്ങളുടെ കര്‍ത്താവായ ദൈവമേ, അങ്ങയുടെ മഹനീയ
ത്രിത്വത്തിന്റെ സംപൂജ്യമായ നാമത്തിനു സ്വര്‍ഗ്ഗത്തിലും ഭൂമിയിലും
എപ്പോഴും സ്തുതിയും ബഹുമാനവും കൃതജ്ഞതയും ആരാധനയും
ഉണ്ടായിരിക്കട്ടെ.
\SignOfCross
}
\Ammen
\end{frame}

% TODO: add prayers according to kalam

\begin{frame}[allowframebreaks]
\frametitle{സങ്കീര്‍ത്തനം}
\Priest{
കര്‍ത്താവേ, മമ രാജാവേ,\\
പാടും നിൻപുകളെന്നും ഞാൻ\\
സകലേശാ, നിൻ തിരുനാമം\\
വാഴ്ത്തീടും ഞാനനവരതം.
}
\framebreak

\People{
കര്‍ത്താവേ, നിൻ സ്‌തുതി പാടും\\
അനുദിനമങ്ങയെ വാഴ്ത്തും ഞാൻ\\
നാഥൻ മഹിമ നിറഞ്ഞവനും\\
പാരം സ്തുത്യനുമെന്നെനും.

അനവദ്യൻ മമ നാഥാ, നിൻ\\
മഹിമയ്ക്കളവില്ലറിവൂ ഞാൻ\\
തലമുറ തലമുറയോടെന്നും\\
തവ കർമ്മങ്ങൾ വിവരിക്കും.
}
\end{frame}

% TODO: add സങ്കീര്‍ത്തനം according to kalam

% not used
\iffalse
\begin{frame}[allowframebreaks]
\frametitle{ധൂപാശീര്‍വ്വാദം}
\Priest{
പിതാവും പുത്രനും പരിശുദ്ധാത്മാവുമായ സര്‍വേശ്വര,
അങ്ങയുടെ ബഹുമാനത്തി നായി ഞങള്‍ സമര്‍പ്പിക്കുന്ന ഈ ധുപം
അങ്ങയുടെ മഹനീയ ത്രിത്വത്തിന്‍റെ നാമത്തില്‍ + ആശീര്‍വ്വദിക്കപ്പെടട്ടെ.
ഇത് അങ്ങയുടെ പ്രസാദത്തിനും അങ്ങയുടെ അജഗണത്തിന്റെ
പാപങ്ങളുടെ മോചനത്തിനും കാരണമാകട്ടെ. എന്നേക്കും.
}
\Ammen
\end{frame}
\fi

\begin{frame}
\Peace
\end{frame}

% Sundays
\begin{frame}[allowframebreaks]
\frametitle{ഉത്ഥാനഗീതത്തിന് ഒരുക്കം}
\Priest{
ഞങ്ങളുടെ കര്‍ത്താവായ ദൈവമേ, അങ്ങയുടെ സ്നേഹത്തിന്റ
പരിമളം ഞങ്ങളില്‍ വീശുകയും അങ്ങയുടെ സത്യത്തിന്റെ
ജ്ഞാനം ഞങ്ങളുടെ ആത്മാക്കളെ പ്രകാശിപ്പിക്കുകയും
ചെയ്യുമ്പോള്‍ സ്വര്‍ഗ്ഗത്തില്‍നിന്നു പ്രത്യക്ഷനാകുന്ന
അങ്ങയുടെ തിരുക്കുമാരനെ സ്വീകരിക്കുവാന്‍ ഞങ്ങള്‍ക്ക്
ഇടയാകട്ടെ. സകല സൗഭാഗ്യങ്ങളും നന്മകളും നിറഞ്ഞ്
മുടിചൂടി നില്ക്കുന്ന സഭയില്‍ നിരന്തരം അങ്ങയെ സ്തുതിക്കുവാനും
മഹത്വപ്പെടുത്തുവാനും ഞങ്ങള്‍ യോഗ്യരാകട്ടെ.
എന്തുകൊണ്ടെന്നാല്‍, അങ്ങ് എല്ലാത്തിന്റെയും സ്രഷ്ടാവാകുന്നു.
സകലത്തിന്റെയും നാഥാ, എന്നേക്കും. 
}
\Ammen
\end{frame}

% Ordinary Days
\begin{frame}[allowframebreaks]
\frametitle{ഉത്ഥാനഗീതത്തിന് ഒരുക്കം}
\Priest{
ഞങ്ങളുടെ കര്‍ത്താവായ ദൈവമേ, അങ്ങു നല്കിയുട്ടുള്ളതും
എന്നാല്‍ കൃത്യജ്ഞത പ്രകാശിപ്പിക്കുവാന്‍ ഞങ്ങള്‍ക്കു
കഴിയാത്തതുമായ എല്ലാ സഹായങ്ങള്‍ക്കും 
അനുഗ്രഹങ്ങള്‍ക്കും ആയി സകല സൗഭാഗ്യങ്ങളും
നന്മകളും നിറഞ്ഞ് മുടിചൂടി നില്‍ക്കുന്ന സഭയില്‍ ഞങ്ങള്‍
അങ്ങയെ നിരന്തരം സ്തുതിക്കുകയും മഹത്വപ്പെടുത്തുകയും ചെയ്യട്ടെ.
അങ്ങ് സകലത്തിന്റെയും നാഥനും സ്രഷ്ടാവുമാകുന്നു.
\SignOfCross
}
\Ammen
\end{frame}


\begin{frame}[allowframebreaks]
\frametitle{ഉത്ഥാനഗീതം}
\People{
സര്‍വാധിപനാം കര്‍ത്താവേ,\\
നിന്നെ വണങ്ങി നമിക്കുന്നു.\\
ഈശോനാഥാ, വിനയമോടെ\\
നിന്നെ നമിച്ചു പുകഴ്ത്തുന്നു.

മർത്യനു നിത്യമഹോന്നതമാം\\
ഉത്ഥാനം നീയരുളുന്നു.\\
അക്ഷയമാവനുടെ ആത്മാവി-\\
ന്നുത്തമരക്ഷയുമേകുന്നു.
}
\end{frame}

\begin{frame}
\LetsPray
\end{frame}

\begin{frame}
\Priest{
എന്റെ കര്‍ത്താവേ, നീ സത്യമായും ഞങ്ങളുടെ ശരീരങ്ങളെ
ഉയര്‍പ്പിക്കുന്നവന്നും ആത്മാക്കളെ രക്ഷിക്കുന്നവനും ജീവനെ
നിത്യം പരിപാലിക്കുന്നവനുമാകുന്നു. ഞങ്ങള്‍ എപ്പോഴും
നിനക്ക് സ്തുതിയും കൃതജ്ഞതയും ആരാധനയും സമര്‍പ്പിക്കുവാന്‍
കടപ്പെട്ടവരാകുന്നു. സകലത്തിന്റെയും നാഥാ, എന്നേക്കും.
}
\Ammen
\end{frame}

\begin{frame}
\frametitle{ത്രൈശുദ്ധ കീര്‍ത്തനം}
\People{
ശബ്ദമുയര്‍ത്തിപ്പാടിടുവിന്‍\\
സര്‍വ്വരൂമൊന്നായി പാടിടുവിന്‍\\
എന്നെന്നും ജീവിക്കും\\
സർവ്വേശ്വരനെ വാഴ്ത്തിടുവിന്‍.

പരിപാവനനാം സര്‍വ്വേശാ,\\
പരിപാവനനാം ബലവാനെ,\\
പരിപാവനനാം അമര്‍ത്യനെ,\\
നിന്‍കൃപ ഞങ്ങള്‍ക്കേകണമേ.
}
\end{frame}

\begin{frame}
\Peace
\end{frame}

\end{document}
